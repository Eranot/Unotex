\chapter{Alíneas}

A ABNT NBR 6024:2012 define que o texto que antecede as alíneas deve terminar com dois pontos; cada alínea deve iniciar com letra minúscula e terminar com ponto e vírgula, exceto a última, finalizada com ponto final, e a alínea que anteceder uma subalínea, finalizada com dois pontos:

\begin{alineas}
  \item item 1;
  \item item 2;
  \item item 3:
  \begin{alineas} % ou \begin{subalineas} ou \begin{incisos}
    \item item 3.1;
    \item item 3.2;
  \end{alineas}
  \item item 4.
\end{alineas}

Note que se a última alínea for uma subalínea, ela será finalizada com ponto final. Além disso, a ABNT define o ponto e vírgula para terminar uma subalínea, enquanto a normalização de trabalhos da UFFS determina o uso da vírgula neste caso. Como desconhecemos se a ABNT define alguma flexibilidade na customização desta regra, deixamos exemplificado o uso da regra original.