\chapter{Citações}

\section{Indiretas}

% Para citar com número de página, basta
% informá-la como parâmetro opcional.
Segundo \textcite{halim3rd}, codar um array de sufixos é mais fácil que uma árvore de sufixos. Ou então: codar um array de sufixos é mais fácil que uma árvore de sufixos~\cite[253]{halim3rd}.

\section{Diretas}

Um dos trechos mais famosos do livro O Pequeno Príncipe é \enquote{O essencial é invisível aos olhos}~\cite[58]{exupery15}.

\subsection{Com mais de três linhas}

\begin{citacao}
\textit{A Léon Werth.}

Peço perdão às crianças por ter dedicado este livro a um adulto. Tenho uma boa desculpa: esse adulto é o meu melhor amigo no mundo. Tenho outra desculpa: esse adulto pode entender tudo, até livros para crianças. Tenho uma terceira desculpa: esse adulto mora na França, onde sente fome e frio. Também precisa ser consolado. Se todas essas desculpas não bastarem, então quero dedicar este livro à criança que esse adulto foi um dia. Todos os adultos primeiro foram crianças. (Mas poucos se lembram disso.) Portanto, corrijo minha dedicatória:

\textit{A Léon Werth quando era pequeno.}~\cite{exupery15}
\end{citacao}