\documentclass[serif, english, brazilian, openany, oneside]{unotex}
\usepackage[utf8]{inputenc}
\usepackage[T1]{fontenc}

\usepackage{threeparttable}

\usepackage[
  style=abnt-numeric,
  style=abnt,
]{biblatex}

\usepackage[useregional=num]{datetime2}
\usepackage{xcolor}
\usepackage{booktabs}
\usepackage[labelfont=bf]{caption}
\usepackage{csquotes}
\usepackage{newtxtext}
\usepackage{newtxmath}
\usepackage{cleveref}
\usepackage{amsmath}
\usepackage{float}
\usepackage{mathtools}
\usepackage{ltablex}
\usepackage{ifthen}

\usepackage{tabularx,ragged2e}
\newcolumntype{C}{>{\arraybackslash}X} %9

\usepackage[
  linesnumbered,
  lined,
  boxed,
  commentsnumbered
]{algorithm2e}

% CONFIGURAÇÃO DO QUADRO
\usepackage{listings}

\lstset{ %
    language=Python,
    basicstyle=\footnotesize,
    numbers=left,
    numberstyle=\tiny\color{gray},
    stepnumber=1,    
    numbersep=5pt,
    backgroundcolor=\color{white},
    showspaces=false,
    showstringspaces=false,
    showtabs=false,
    frame=single,
    rulecolor=\color{black},
    tabsize=2,
    captionpos=t,
    breaklines=true,
    breakatwhitespace=false,
    escapeinside={\%*}{*)},
    morekeywords={import,from,class,def,for,while,if,is,in,elif,else,not,and,or,print,break,continue,return,True,False,None,access,as,,del,except,exec,finally,global,import,lambda,pass,print,raise,try,assert}
}

% et al com itálico
\renewbibmacro*{name:andothers}{
  \ifboolexpr{
    test {\ifnumequal{\value{listcount}}{\value{liststop}}}
    and
    test \ifmorenames
  }
    {\ifnumgreater{\value{liststop}}{1}
       {\finalandcomma}
       {}%
     \andothersdelim\bibstring[\emph]{andothers}}
    {}}


% Adiciona arquivo de referências para o biblatex
\addbibresource{referencias.bib}

% Comandos de dados
\autor{Renato Mello Konflanz}
\instituicao{Universidade Comunitária da Região de Chapecó}
\newcommand{\unochapeco}{(Unochapecó)}
\newcommand{\dataeano}{Chapecó - SC, 2019}
\newcommand{\dataemeseano}{Chapecó, Dezembro de 2019}
\newcommand{\dataediaemeseano}{Chapecó, 04 de Dezembro de 2019}
\newcommand{\area}{Área de ciências exatas e ambientais}
\newcommand{\nomeorientador}{Prof. Marcos Antonio Moretto}
\newcommand{\tituloorientador}{M.Sc}
\newcommand{\tipocurso}{Bacharelado}

\newcommand{\titulodoautor}{Bacharel em Sistemas de Informação}

\newcommand{\nomebancaum}{Prof. Giancarlo Salton}
\newcommand{\titulobancaum}{Dr.}

\newcommand{\nomebancadois}{Prof. Sandro Oliveira}
\newcommand{\titulobancadois}{MSc.}

\newcommand{\nomesupervisor}{Prof. Sandro Oliveira}
\newcommand{\titulosupervisor}{MSc.}

\newcommand{\nomecoord}{Profa. Mônica Tissiani de Toni Pereira}
\newcommand{\titulocoord}{MSc.}

\titulo{Título do seu trabalho de conclusão de curso}
%\subtitulo{Um subtitulo se você tiver}
\curso{Sistemas de Informação}
\uf{SC}
\renewcommand{\preambulo}{Relatório do Trabalho de Conclusão de Curso submetido à Universidade Comunitária da Região de Chapecó para obtenção do título de bacharelado no curso de Sistemas de Informação.}
\data{\hoje}
% \coorientador{Beltrano da Computação}
\tipotrabalho{Trabalho de conclusão de curso (graduação)}

\begin{document}
\pretextual%

\imprimircapa
\imprimirfolhaderosto

\imprimirfolhadeaprovacao
{
    \vspace*{19cm}
    \hspace*{7cm}
    \begin{minipage}{0.5\textwidth}
        \mdseries
        \SingleSpacing
    
        Lorem ipsum dolor sit amet, consectetur adipiscing elit. Maecenas in quam placerat, accumsan est eget, placerat ligula. Nullam eget neque feugiat, lobortis dolor sit amet.
    \end{minipage}
    \vspace{1cm}
}

\cleardoublepage


\pdfbookmark{\listfigurename}{lof}
\listoffigures*
\cleardoublepage%

\listofequations*
\cleardoublepage

\pdfbookmark[0]{\listofquadrosname}{loq}
\listofquadros*
\cleardoublepage

\listoftables*
\cleardoublepage

\begin{resumo}
  Lorem ipsum dolor sit amet, consectetur adipiscing elit. Maecenas in quam placerat, accumsan est eget, placerat ligula. Nullam eget neque feugiat, lobortis dolor sit amet, molestie turpis. Donec et turpis hendrerit, eleifend ex quis, mollis ante. Nulla facilisi. Pellentesque habitant morbi tristique senectus et netus et malesuada fames ac turpis egestas. Fusce pellentesque lorem urna, vitae ultricies lorem cursus et. Ut convallis ante vel dui sodales, et vulputate massa viverra. Aenean et tincidunt sapien, non finibus diam. Quisque vulputate metus sed vehicula consectetur. Donec ut ex ut arcu tempor pharetra ut id nibh. Nulla ac fringilla ligula, vel dapibus sapien. Morbi bibendum consectetur dignissim. Praesent laoreet eleifend elementum. Quisque ut hendrerit orci, vitae fermentum metus. In pharetra ex purus, eu scelerisque felis finibus ac. Vivamus vel odio convallis, porta elit vel, laoreet orci. Lorem ipsum dolor sit amet, consectetur adipiscing elit. Maecenas in quam placerat, accumsan est eget, placerat ligula. Nullam eget neque feugiat, lobortis dolor sit amet, molestie turpis. Donec et turpis hendrerit, eleifend ex quis, mollis ante. Nulla facilisi. Pellentesque habitant morbi tristique senectus et netus et malesuada fames ac turpis egestas. Fusce pellentesque lorem urna, vitae ultricies lorem cursus et. Ut convallis ante vel dui sodales, et vulputate massa viverra. Aenean et tincidunt sapien, non finibus diam. Quisque vulputate metus sed vehicula consectetur. Donec ut ex ut arcu tempor pharetra ut id nibh. Nulla ac fringilla ligula, vel dapibus sapien. Morbi bibendum consectetur dignissim. Praesent laoreet eleifend elementum. Quisque ut hendrerit orci, vitae fermentum metus. In pharetra ex purus, eu scelerisque felis finibus ac. Vivamus vel odio convallis, porta elit vel, laoreet orci.

  \vspace{\baselineskip}

  Palavras-chave: Palavra 1; Palavra 2; Palavra 3.
\end{resumo}

\begin{resumo}[Abstract]
  \begin{otherlanguage*}{english}

    Lorem ipsum dolor sit amet, consectetur adipiscing elit. Maecenas in quam placerat, accumsan est eget, placerat ligula. Nullam eget neque feugiat, lobortis dolor sit amet, molestie turpis. Donec et turpis hendrerit, eleifend ex quis, mollis ante. Nulla facilisi. Pellentesque habitant morbi tristique senectus et netus et malesuada fames ac turpis egestas. Fusce pellentesque lorem urna, vitae ultricies lorem cursus et. Ut convallis ante vel dui sodales, et vulputate massa viverra. Aenean et tincidunt sapien, non finibus diam. Quisque vulputate metus sed vehicula consectetur. Donec ut ex ut arcu tempor pharetra ut id nibh. Nulla ac fringilla ligula, vel dapibus sapien. Morbi bibendum consectetur dignissim. Praesent laoreet eleifend elementum. Quisque ut hendrerit orci, vitae fermentum metus. In pharetra ex purus, eu scelerisque felis finibus ac. Vivamus vel odio convallis, porta elit vel, laoreet orci. Lorem ipsum dolor sit amet, consectetur adipiscing elit. Maecenas in quam placerat, accumsan est eget, placerat ligula. Nullam eget neque feugiat, lobortis dolor sit amet, molestie turpis. Donec et turpis hendrerit, eleifend ex quis, mollis ante. Nulla facilisi. Pellentesque habitant morbi tristique senectus et netus et malesuada fames ac turpis egestas. Fusce pellentesque lorem urna, vitae ultricies lorem cursus et. Ut convallis ante vel dui sodales, et vulputate massa viverra. Aenean et tincidunt sapien, non finibus diam. Quisque vulputate metus sed vehicula consectetur. Donec ut ex ut arcu tempor pharetra ut id nibh. Nulla ac fringilla ligula, vel dapibus sapien. Morbi bibendum consectetur dignissim. Praesent laoreet eleifend elementum. Quisque ut hendrerit orci, vitae fermentum metus. In pharetra ex purus, eu scelerisque felis finibus ac. Vivamus vel odio convallis, porta elit vel, laoreet orci.
    
    \vspace{\baselineskip}

    Keywords: Word 1; Word 2; Word 3.
  \end{otherlanguage*}
\end{resumo}

% Coloca o sumário no "sumário do PDF"
% (chamado de bookmark do PDF).
\pdfbookmark{\contentsname}{toc}
% Versão com * não coloca o próprio sumário no sumário
\tableofcontents*
\cleardoublepage%

% Início da parte textual
\textual%

\chapter{Introdução}
 
\section{Contextualização}

Lorem ipsum dolor sit amet, consectetur adipiscing elit. In volutpat fringilla libero vel dictum. Praesent tempus blandit sem, a convallis nisi placerat ut. Maecenas aliquam nibh nec neque placerat euismod. Suspendisse viverra libero vitae mi tincidunt, eget mollis dui malesuada. Vivamus malesuada nibh lacus, ut semper est semper vel. Proin sed pretium augue. Suspendisse vulputate ligula sed viverra scelerisque. Suspendisse varius placerat ex, quis ullamcorper arcu ultrices non. Duis convallis diam vel nisl interdum, eu sollicitudin nibh vehicula.

Lorem ipsum dolor sit amet, consectetur adipiscing elit. In volutpat fringilla libero vel dictum. Praesent tempus blandit sem, a convallis nisi placerat ut. Maecenas aliquam nibh nec neque placerat euismod. Suspendisse viverra libero vitae mi tincidunt, eget mollis dui malesuada. Vivamus malesuada nibh lacus, ut semper est semper vel. Proin sed pretium augue. Suspendisse vulputate ligula sed viverra scelerisque. Suspendisse varius placerat ex, quis ullamcorper arcu ultrices non. Duis convallis diam vel nisl interdum, eu sollicitudin nibh vehicula.

\section{Delimitação do problema}

Lorem ipsum dolor sit amet, consectetur adipiscing elit. In volutpat fringilla libero vel dictum. Praesent tempus blandit sem, a convallis nisi placerat ut. Maecenas aliquam nibh nec neque placerat euismod. Suspendisse viverra libero vitae mi tincidunt, eget mollis dui malesuada. Vivamus malesuada nibh lacus, ut semper est semper vel. Proin sed pretium augue. Suspendisse vulputate ligula sed viverra scelerisque. Suspendisse varius placerat ex, quis ullamcorper arcu ultrices non. Duis convallis diam vel nisl interdum, eu sollicitudin nibh vehicula.

\section{Objetivos}

\subsection{Objetivo geral}

Lorem ipsum dolor sit amet, consectetur adipiscing elit. In volutpat fringilla libero vel dictum. Praesent tempus blandit sem, a convallis nisi placerat ut. Maecenas aliquam nibh nec neque placerat euismod. Suspendisse viverra libero vitae mi tincidunt, eget mollis dui malesuada. Vivamus malesuada nibh lacus, ut semper est semper vel. Proin sed pretium augue. Suspendisse vulputate ligula sed viverra scelerisque. Suspendisse varius placerat ex, quis ullamcorper arcu ultrices non. Duis convallis diam vel nisl interdum, eu sollicitudin nibh vehicula.

\subsection{Objetivos específicos}

\begin{alineas}
  \item Lorem ipsum dolor sit amet, consectetur adipiscing elit;
  \item Lorem ipsum dolor sit amet, consectetur adipiscing elit;
  \item Lorem ipsum dolor sit amet, consectetur adipiscing elit;
  \item Lorem ipsum dolor sit amet, consectetur adipiscing elit.
 \end{alineas}





\section{Justificativa}

Lorem ipsum dolor sit amet, consectetur adipiscing elit. Cras fermentum pretium leo, non eleifend massa ornare vitae. Vestibulum iaculis sapien sit amet ex aliquet, sit amet porta velit vulputate. In leo nisi, hendrerit sit amet tempus eu, ornare et neque. Sed ultrices, velit id sagittis mattis, nisi sem cursus justo, a lobortis justo mauris quis risus. Etiam ut sagittis massa. Morbi laoreet, tellus non sollicitudin dignissim, leo lorem varius mi, ac maximus nisl sem id nunc. Proin lorem massa, bibendum eget pellentesque eget, porttitor eu urna.

Lorem ipsum dolor sit amet, consectetur adipiscing elit. Cras fermentum pretium leo, non eleifend massa ornare vitae. Vestibulum iaculis sapien sit amet ex aliquet, sit amet porta velit vulputate. In leo nisi, hendrerit sit amet tempus eu, ornare et neque. Sed ultrices, velit id sagittis mattis, nisi sem cursus justo, a lobortis justo mauris quis risus. Etiam ut sagittis massa. Morbi laoreet, tellus non sollicitudin dignissim, leo lorem varius mi, ac maximus nisl sem id nunc. Proin lorem massa, bibendum eget pellentesque eget, porttitor eu urna.

Lorem ipsum dolor sit amet, consectetur adipiscing elit. Cras fermentum pretium leo, non eleifend massa ornare vitae. Vestibulum iaculis sapien sit amet ex aliquet, sit amet porta velit vulputate. In leo nisi, hendrerit sit amet tempus eu, ornare et neque. Sed ultrices, velit id sagittis mattis, nisi sem cursus justo, a lobortis justo mauris quis risus. Etiam ut sagittis massa. Morbi laoreet, tellus non sollicitudin dignissim, leo lorem varius mi, ac maximus nisl sem id nunc. Proin lorem massa, bibendum eget pellentesque eget, porttitor eu urna.

Lorem ipsum dolor sit amet, consectetur adipiscing elit. Cras fermentum pretium leo, non eleifend massa ornare vitae. Vestibulum iaculis sapien sit amet ex aliquet, sit amet porta velit vulputate. In leo nisi, hendrerit sit amet tempus eu, ornare et neque. Sed ultrices, velit id sagittis mattis, nisi sem cursus justo, a lobortis justo mauris quis risus. Etiam ut sagittis massa. Morbi laoreet, tellus non sollicitudin dignissim, leo lorem varius mi, ac maximus nisl sem id nunc. Proin lorem massa, bibendum eget pellentesque eget, porttitor eu urna.

Lorem ipsum dolor sit amet, consectetur adipiscing elit. Cras fermentum pretium leo, non eleifend massa ornare vitae. Vestibulum iaculis sapien sit amet ex aliquet, sit amet porta velit vulputate. In leo nisi, hendrerit sit amet tempus eu, ornare et neque. Sed ultrices, velit id sagittis mattis, nisi sem cursus justo, a lobortis justo mauris quis risus. Etiam ut sagittis massa. Morbi laoreet, tellus non sollicitudin dignissim, leo lorem varius mi, ac maximus nisl sem id nunc. Proin lorem massa, bibendum eget pellentesque eget, porttitor eu urna. Lorem ipsum dolor sit amet, consectetur adipiscing elit. Cras fermentum pretium leo, non eleifend massa ornare vitae. Vestibulum iaculis sapien sit amet ex aliquet, sit amet porta velit vulputate. In leo nisi, hendrerit sit amet tempus eu, ornare et neque. Sed ultrices, velit id sagittis mattis, nisi sem cursus justo, a lobortis justo mauris quis risus. Etiam ut sagittis massa. Morbi laoreet, tellus non sollicitudin dignissim, leo lorem varius mi, ac maximus nisl sem id nunc. Proin lorem massa, bibendum eget pellentesque eget, porttitor eu urna.

\section{Procedimentos Metodológicos}

Lorem ipsum dolor sit amet, consectetur adipiscing elit. In volutpat fringilla libero vel dictum. Praesent tempus blandit sem, a convallis nisi placerat ut. Maecenas aliquam nibh nec neque placerat euismod. Suspendisse viverra libero vitae mi tincidunt, eget mollis dui malesuada. Vivamus malesuada nibh lacus, ut semper est semper vel. Proin sed pretium augue. Suspendisse vulputate ligula sed viverra scelerisque. Suspendisse varius placerat ex, quis ullamcorper arcu ultrices non. Duis convallis diam vel nisl interdum, eu sollicitudin nibh vehicula.

\section{Estrutura do trabalho}

Lorem ipsum dolor sit amet, consectetur adipiscing elit. In volutpat fringilla libero vel dictum. Praesent tempus blandit sem, a convallis nisi placerat ut. Maecenas aliquam nibh nec neque placerat euismod. Suspendisse viverra libero vitae mi tincidunt, eget mollis dui malesuada. Vivamus malesuada nibh lacus, ut semper est semper vel. Proin sed pretium augue. Suspendisse vulputate ligula sed viverra scelerisque. Suspendisse varius placerat ex, quis ullamcorper arcu ultrices non. Duis convallis diam vel nisl interdum, eu sollicitudin nibh vehicula.

Lorem ipsum dolor sit amet, consectetur adipiscing elit. In volutpat fringilla libero vel dictum. Praesent tempus blandit sem, a convallis nisi placerat ut. Maecenas aliquam nibh nec neque placerat euismod. Suspendisse viverra libero vitae mi tincidunt, eget mollis dui malesuada. Vivamus malesuada nibh lacus, ut semper est semper vel. Proin sed pretium augue. Suspendisse vulputate ligula sed viverra scelerisque. Suspendisse varius placerat ex, quis ullamcorper arcu ultrices non. Duis convallis diam vel nisl interdum, eu sollicitudin nibh vehicula.
\chapter{Citações}

Lorem ipsum dolor sit amet, consectetur adipiscing elit. Cras fermentum pretium leo, non eleifend massa ornare vitae. Vestibulum iaculis sapien sit amet ex aliquet, sit amet porta velit vulputate. In leo nisi, hendrerit sit amet tempus eu, ornare et neque. Sed ultrices, velit id sagittis mattis, nisi sem cursus justo, a lobortis justo mauris quis risus. Etiam ut sagittis massa. Morbi laoreet, tellus non sollicitudin dignissim, leo lorem varius mi, ac maximus nisl sem id nunc. Proin lorem massa, bibendum eget pellentesque eget, porttitor eu urna. \enquote{Morbi laoreet, tellus non sollicitudin dignissim, leo lorem varius mi, ac maximus nisl sem id nunc. Proin lorem massa, bibendum eget pellentesque eget, porttitor eu urna} \cite{netto}.

Lorem ipsum dolor sit amet, \textcite{netto} morbi laoreet, tellus non sollicitudin dignissim, leo lorem varius mi, ac maximus nisl sem id nunc. Proin lorem massa, bibendum eget pellentesque eget, porttitor eu urna. 
\begin{citacao}
Lorem ipsum dolor sit amet, consectetur adipiscing elit. Cras fermentum pretium leo, non eleifend massa ornare vitae. Vestibulum iaculis sapien sit amet ex aliquet, sit amet porta velit vulputate. In leo nisi, hendrerit sit amet tempus eu, ornare et neque. Sed ultrices, velit id sagittis mattis, nisi sem cursus justo, a lobortis justo mauris quis risus. Etiam ut sagittis massa. Morbi laoreet, tellus non sollicitudin dignissim, leo lorem varius mi, ac maximus nisl sem id nunc. \cite[p. 5]{gonzalez}.
\end{citacao}

Lorem ipsum dolor sit amet, consectetur adipiscing elit. Cras fermentum pretium leo, non eleifend massa ornare vitae. Vestibulum iaculis sapien sit amet ex aliquet, sit amet porta velit vulputate. In leo nisi, hendrerit sit amet tempus eu, ornare et neque. Sed ultrices, velit id sagittis mattis, nisi sem cursus justo, a lobortis justo mauris quis risus. Etiam ut sagittis massa.


\chapter{Imagens}

Lorem ipsum dolor sit amet, consectetur adipiscing elit. Cras fermentum pretium leo, non eleifend massa ornare vitae. Vestibulum iaculis sapien sit amet ex aliquet, sit amet porta velit vulputate. In leo nisi, hendrerit sit amet tempus eu, ornare et neque. Sed ultrices, velit id sagittis mattis, nisi sem cursus justo, a lobortis justo mauris quis risus. Etiam ut sagittis massa. Morbi laoreet, tellus non sollicitudin dignissim, leo lorem varius mi, ac maximus nisl sem id nunc. Proin lorem massa, bibendum eget pellentesque eget, porttitor eu urna \autoref{fig:konigsberg}.

\fig{Pontes de Königsberg}{konigsberg}{scale=0.6}{img/konigsberg.png}{\cite{netto}}{ht}

Lorem ipsum dolor sit amet, consectetur adipiscing elit. Cras fermentum pretium leo, non eleifend massa ornare vitae. Vestibulum iaculis sapien sit amet ex aliquet, sit amet porta velit vulputate. In leo nisi, hendrerit sit amet tempus eu, ornare et neque. Sed ultrices, velit id sagittis mattis, nisi sem cursus justo, a lobortis justo mauris quis risus. Etiam ut sagittis massa. Morbi laoreet, tellus non sollicitudin dignissim, leo lorem varius mi, ac maximus nisl sem id nunc. Proin lorem massa, bibendum eget pellentesque eget, porttitor eu urna.


\fig{Vértice adjacentes e laço}{adjacente_laco}{scale=0.6}{img/adjacente_laco.png}{elaboração do autor}{ht}

Lorem ipsum dolor sit amet, \autoref{fig:adjacente_laco} consectetur adipiscing elit. Cras fermentum pretium leo, non eleifend massa ornare vitae. Vestibulum iaculis sapien sit amet ex aliquet, sit amet porta velit vulputate. In leo nisi, hendrerit sit amet tempus eu, ornare et neque. Sed ultrices, velit id sagittis mattis, nisi sem cursus justo, a lobortis justo mauris quis risus. Etiam ut sagittis massa. Morbi laoreet, tellus non sollicitudin dignissim, leo lorem varius mi, ac maximus nisl sem id nunc. Proin lorem massa, bibendum eget pellentesque eget, porttitor eu urna.


\fig{Exemplos de imagens filtradas por filtros de média}{exemplos_filtros_media}{scale=0.4}{img/exemplos_filtros_media.png}{\textcite{gonzalez}}{ht}

\chapter{Equações}

Lorem ipsum dolor sit amet, consectetur adipiscing elit. Cras fermentum pretium leo, non eleifend massa ornare vitae. Vestibulum iaculis sapien sit amet ex aliquet, sit amet porta velit vulputate. In leo nisi, hendrerit sit amet tempus eu, ornare et neque. Sed ultrices, velit id sagittis mattis, nisi sem cursus justo, a lobortis justo mauris quis risus. Etiam ut sagittis massa. Morbi laoreet, tellus non sollicitudin dignissim, leo lorem varius mi, ac maximus nisl sem id nunc. Proin lorem massa, bibendum eget pellentesque eget, porttitor eu urna.


\eq{ht}{Equação de vizinhança de um vértice}{vizinhanca}{
 {} & V(v) = \{w \in V|vw \in A\}\\
    &\footnotesize\text{Fonte: \textcite{costa}}
}

Lorem ipsum dolor sit amet, consectetur adipiscing elit. Cras fermentum pretium leo, non eleifend massa ornare vitae. Vestibulum iaculis sapien sit amet ex aliquet, sit amet porta velit vulputate \autoref{eq:vizinhanca}.

\eq{ht}{Função de normalização de intensidade}{eq_normalization}{
 {} & p_n = \dfrac{\text{número de \textit{pixels} com intensidade n}}{\text{número total de \textit{pixels}}}\\
    &\footnotesize\text{Fonte: \textcite{uci}}
}

Lorem ipsum dolor sit amet \autoref{eq:eq_normalization}, consectetur adipiscing elit. Cras fermentum pretium leo, non eleifend massa ornare vitae. Vestibulum iaculis sapien sit amet ex aliquet, sit amet porta velit vulputate.

\eq{ht}{Função da equalização de histograma}{histogram_eq}{
 {} & g_{i, j} = floor((L - 1) \sum_{n=0}^{f_{i, j}} p_n)\\
    &\footnotesize\text{Fonte: \textcite{uci}}
}

Lorem ipsum dolor sit amet \autoref{eq:histogram_eq}, consectetur adipiscing elit. Cras fermentum pretium leo, non eleifend massa ornare vitae. Vestibulum iaculis sapien sit amet ex aliquet, sit amet porta velit vulputate.

\chapter{Quadros de código}

Lorem ipsum dolor sit amet, consectetur adipiscing elit. Cras fermentum pretium leo, non eleifend massa ornare vitae. Vestibulum iaculis sapien sit amet ex aliquet, sit amet porta velit vulputate. In leo nisi, hendrerit sit amet tempus eu, ornare et neque. Sed ultrices, velit id sagittis mattis, nisi sem cursus justo, a lobortis justo mauris quis risus. Etiam ut sagittis massa. Morbi laoreet, tellus non sollicitudin dignissim, leo lorem varius mi, ac maximus nisl sem id nunc. Proin lorem massa, bibendum eget pellentesque eget, porttitor eu urna. \autoref{load_image} lorem ipsum dolor sit amet.

\begin{quadro}[H]
    \caption{Exemplo de leitura e visualização de imagem}
    \label{load_image}
    \begin{scriptsize}
    \lstinputlisting[language=Python]{codes/load_image.py}
    \end{scriptsize}
    \raggedright\footnotesize\text{Fonte: elaboração do autor}
\end{quadro}

\chapter{Tabelas}

Lorem ipsum dolor sit amet, consectetur adipiscing elit. Cras fermentum pretium leo, non eleifend massa ornare vitae. Vestibulum iaculis sapien sit amet ex aliquet, sit amet porta velit vulputate. In leo nisi, hendrerit sit amet tempus eu, ornare et neque. Sed ultrices, velit id sagittis mattis, nisi sem cursus justo, a lobortis justo mauris quis risus. Etiam ut sagittis massa. Morbi laoreet, tellus non sollicitudin dignissim, leo lorem varius mi, ac maximus nisl sem id nunc.

\begin{table}[h!]\begin{center}
	\caption{Cronograma}\label{tab-cronograma}
	\begin{tabular*}{\textwidth}{@{\extracolsep{\fill}} c c c c c c c c c c c c c}
		\toprule
		& Etapa & Fev & Mar & Abr & Mai & Jun & Jul & Ago & Set & Out & Nov &\\
		\midrule
		&   1   &  x  &  x  &  x  &  -  &  -  &  -  &  -  &  -  &  -  &  -  &\\
		&   2   &  x  &  x  &  -  &  -  &  -  &  -  &  -  &  -  &  -  &  -  &\\
		&   3   &  x  &  x  &  x  &  x  &  x  &  x  &  x  &  -  &  -  &  -  &\\
		&   4   &  -  &  -  &  -  &  x  &  x  &  -  &  -  &  -  &  x  &  x  &\\
		&   5   &  -  &  -  &  -  &  -  &  -  &  x  &  -  &  -  &  -  &  -  &\\
		&   6   &  -  &  -  &  -  &  -  &  -  &  x  &  x  &  -  &  -  &  -  &\\
		&   7   &  -  &  -  &  -  &  -  &  -  &  x  &  x  &  x  &  -  &  -  &\\
		&   8   &  -  &  -  &  -  &  -  &  -  &  -  &  x  &  x  &  x  &  -  &\\
		&   9   &  -  &  -  &  -  &  -  &  -  &  -  &  -  &  -  &  x  &  x  &\\
		&   10  &  -  &  -  &  -  &  -  &  -  &  -  &  -  &  -  &  -  &  x  &\\
		\bottomrule   
	\end{tabular*}
\end{center}\end{table}

\begin{table}[h!]\begin{center}
	\caption{Orçamento}\label{tab-orcamento}
	\begin{tabular*}{\textwidth}{@{\extracolsep{\fill}} l l c l c}
		\toprule
		& Item  & Quantidade & Custo unitário &\\
		\midrule
		& Impressão e encadernação espiral   &  3  &  R\$ 15,00  &\\
		& Carcaça do robô com 4 motores   &  1  &  R\$ 500,00  &\\
		& Raspberry Pi 3 Model B+   &  1  &  R\$ 290,00  &\\
		& Câmera \textit{webcam}   &  1  &  R\$ 100,00  &\\
		& Ponte H & 2 & R\$ 13,00&\\
		& Bateria   &  2  &  R\$ 90,00  &\\
		& Lâmpada   &  1  &  R\$ 35,00  &\\
		& Sensor ultrassônico de proximidade   &  1  &  R\$ 8,00  &\\
		\bottomrule
		& & & Valor total: R\$ 1184,00  &
	\end{tabular*}
\end{center}\end{table}

\chapter{Outros}

\section{Listas}

In leo nisi, hendrerit sit amet tempus eu, ornare et neque. Sed ultrices, velit id sagittis mattis, nisi sem cursus justo, a lobortis justo mauris quis risus. Etiam ut sagittis massa.

\begin{enumerate}
    \item Detecção de pisos táteis;
    \item Detecção de intersecções em pisos táteis;
    \item Uma forma de mapear ambientes;
    \item Utilização da detecção do pisos táteis e do mapa para criar um direcionador que aponta o caminho entre dois locais do mapa.
\end{enumerate}


\section{Aspas}

Lorem ipsum dolor sit amet, \enquote{consectetur adipiscing elit. Cras fermentum pretium leo, non eleifend massa ornare} vitae. Vestibulum iaculis sapien sit amet ex aliquet, sit amet porta velit vulputate. In leo nisi, hendrerit sit amet tempus eu, ornare et neque. Sed ultrices, velit id sagittis mattis, nisi sem cursus justo, a lobortis justo mauris quis risus. Etiam ut sagittis massa. Morbi laoreet, tellus non sollicitudin dignissim, leo lorem varius mi, ac maximus nisl sem id nunc. Proin lorem massa, bibendum eget pellentesque eget, porttitor eu urna.

\section{Itálico}

In leo nisi, \textit{hendrerit sit amet tempus eu, ornare et neque}. Sed ultrices, velit id sagittis mattis, nisi sem cursus justo, a lobortis justo mauris quis risus. Etiam ut sagittis massa.
\chapter{Conclusões}

Lorem ipsum dolor sit amet, consectetur adipiscing elit. Vivamus pellentesque quam non aliquet suscipit. Nullam molestie augue dolor. Sed quis lectus lectus. Pellentesque habitant morbi tristique senectus et netus et malesuada fames ac turpis egestas. Cras eget erat ullamcorper, accumsan diam vitae, blandit tellus. Donec vel eleifend dolor, sed vestibulum odio. In elementum purus at erat consequat, sed congue mi dignissim. Pellentesque eget nulla id diam venenatis tristique. Curabitur nec sem turpis. Phasellus sed porttitor turpis. Aliquam a dolor eget odio vulputate faucibus non sed dui.

Nullam tempor leo massa, nec dictum nunc pharetra vitae. Cras vel ipsum sit amet ex aliquet fermentum sit amet nec erat. Ut ac est vitae ex aliquam lacinia eu vitae dolor. Ut ornare commodo sapien, nec tincidunt nulla posuere a. Quisque a magna vitae nulla luctus placerat a in augue. Fusce iaculis hendrerit arcu. Orci varius natoque penatibus et magnis dis parturient montes, nascetur ridiculus mus. Orci varius natoque penatibus et magnis dis parturient montes, nascetur ridiculus mus. Donec ut posuere felis. In hac habitasse platea dictumst. Fusce consectetur vel odio sed fringilla. Cras sit amet turpis eleifend, vulputate eros ut, varius lacus.



\section{Trabalhos futuros}

Aliquam erat volutpat. Cras justo nisl, dignissim vitae sollicitudin tincidunt, feugiat in massa. Pellentesque finibus massa sapien, nec volutpat quam ullamcorper vel. Cras ultrices felis tempus, ultrices leo vel, convallis turpis. Maecenas ac lobortis metus, et luctus risus. Vivamus condimentum eleifend augue, nec maximus tellus sagittis non. Vivamus quis odio sed diam dictum efficitur eget eget felis. Integer scelerisque dolor nec risus accumsan maximus. Pellentesque iaculis lorem in semper maximus. Mauris dignissim sit amet risus volutpat sollicitudin. Donec in sem dui. Duis at fermentum velit. Integer placerat nunc eget sem gravida, a sodales leo condimentum.

\postextual%

\printbibliography[heading=abnt]

\end{document}